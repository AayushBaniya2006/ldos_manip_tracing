% =============================================================================
% LDOS Research Report Template
% Tracing-Driven Performance Analysis for ROS 2 Manipulation Stacks
% =============================================================================

\documentclass[11pt,a4paper]{article}

% =============================================================================
% PACKAGES
% =============================================================================

% Layout and formatting
\usepackage[margin=1in]{geometry}
\usepackage{setspace}
\usepackage{parskip}

% Math and symbols
\usepackage{amsmath,amssymb,amsfonts}

% Graphics and figures
\usepackage{graphicx}
\usepackage{float}
\usepackage{subcaption}
\usepackage{tikz}

% Tables
\usepackage{booktabs}
\usepackage{multirow}
\usepackage{longtable}
\usepackage{tabularx}

% Code listings
\usepackage{listings}
\usepackage{xcolor}

% References and links
\usepackage{hyperref}
\usepackage{cleveref}

% Bibliography
\usepackage[backend=biber,style=ieee]{biblatex}
% \addbibresource{references.bib}

% =============================================================================
% CONFIGURATION
% =============================================================================

% Colors
\definecolor{codegreen}{rgb}{0,0.6,0}
\definecolor{codegray}{rgb}{0.5,0.5,0.5}
\definecolor{codepurple}{rgb}{0.58,0,0.82}
\definecolor{backcolour}{rgb}{0.95,0.95,0.92}

% Code listing style
\lstdefinestyle{mystyle}{
    backgroundcolor=\color{backcolour},
    commentstyle=\color{codegreen},
    keywordstyle=\color{blue},
    numberstyle=\tiny\color{codegray},
    stringstyle=\color{codepurple},
    basicstyle=\ttfamily\footnotesize,
    breakatwhitespace=false,
    breaklines=true,
    captionpos=b,
    keepspaces=true,
    numbers=left,
    numbersep=5pt,
    showspaces=false,
    showstringspaces=false,
    showtabs=false,
    tabsize=2
}
\lstset{style=mystyle}

% Hyperref configuration
\hypersetup{
    colorlinks=true,
    linkcolor=blue,
    filecolor=magenta,
    urlcolor=cyan,
    citecolor=green,
    pdftitle={LDOS Research Report},
    pdfauthor={Author Name}
}

% Custom commands
\newcommand{\ros}{ROS~2\xspace}
\newcommand{\moveit}{MoveIt~2\xspace}
\newcommand{\gazebo}{Gazebo\xspace}

% =============================================================================
% DOCUMENT
% =============================================================================

\begin{document}

% -----------------------------------------------------------------------------
% Title Page
% -----------------------------------------------------------------------------

\begin{titlepage}
    \centering
    \vspace*{2cm}

    {\LARGE\bfseries Tracing-Driven Performance Analysis for\\ROS 2 Manipulation Stacks\par}

    \vspace{1.5cm}

    {\Large Undergraduate Research Project\par}

    \vspace{2cm}

    {\large
        \textbf{Author:} Your Name\\[0.3cm]
        \textbf{Mentor:} Rohit Dwivedula\\[0.3cm]
        \textbf{Institution:} Your University\\[0.3cm]
    }

    \vspace{2cm}

    {\large \today\par}

    \vfill

    % Optional: Add university logo
    % \includegraphics[width=0.3\textwidth]{logo.png}

\end{titlepage}

% -----------------------------------------------------------------------------
% Abstract
% -----------------------------------------------------------------------------

\begin{abstract}
This report presents a comprehensive performance analysis of \ros manipulation stacks under various load conditions using low-overhead tracing. We developed an automated experiment harness that combines LTTng-based tracing with controlled load generation to characterize end-to-end latencies in motion planning and execution pipelines.

Our experiments on CloudLab bare-metal nodes reveal [key findings]. We identify critical callback chains that become bottlenecks under CPU stress and message flooding scenarios. Statistical analysis using [methods] demonstrates significant latency increases of [X\%] under [conditions].

The results provide actionable insights for optimizing real-time performance in robotic manipulation systems and establish a reproducible methodology for timing analysis in \ros applications.

\textbf{Keywords:} ROS 2, performance analysis, tracing, manipulation, MoveIt, real-time systems
\end{abstract}

\tableofcontents
\newpage

% -----------------------------------------------------------------------------
% Introduction
% -----------------------------------------------------------------------------

\section{Introduction}
\label{sec:introduction}

\subsection{Motivation}

Robotic manipulation systems require predictable timing behavior for safe and effective operation. The \ros ecosystem provides powerful tools for building complex robotic applications, but understanding end-to-end latencies in these systems remains challenging.

\subsection{Research Questions}

This work addresses the following research questions:

\begin{enumerate}
    \item How do end-to-end latencies in \ros manipulation stacks vary under different load conditions?
    \item Which callback chains are most sensitive to CPU and message load?
    \item What are the breaking points beyond which the system fails to meet timing requirements?
\end{enumerate}

\subsection{Contributions}

The main contributions of this work are:

\begin{itemize}
    \item An automated experiment harness for reproducible performance analysis
    \item Comprehensive latency characterization under baseline and load scenarios
    \item Identification of bottleneck callbacks and critical paths
    \item Statistical validation of performance degradation patterns
\end{itemize}

% -----------------------------------------------------------------------------
% Background
% -----------------------------------------------------------------------------

\section{Background}
\label{sec:background}

\subsection{ROS 2 Architecture}

\ros is a middleware framework for robotics that provides...

\subsection{MoveIt 2}

\moveit is the primary motion planning framework for \ros...

\subsection{LTTng Tracing}

LTTng (Linux Trace Toolkit: next generation) provides low-overhead tracing...

\subsection{Related Work}

% Cite PiCAS, PAAM, and other relevant papers
Previous work on \ros performance analysis includes...

% -----------------------------------------------------------------------------
% Methodology
% -----------------------------------------------------------------------------

\section{Methodology}
\label{sec:methodology}

\subsection{Experiment Setup}

% Include hardware table
\input{analysis/output/hardware_info.tex}

\subsubsection{Software Configuration}

\begin{itemize}
    \item \textbf{ROS 2 Distribution:} Jazzy Jalisco
    \item \textbf{Simulator:} Gazebo Harmonic (gz-sim8)
    \item \textbf{Robot:} Franka Emika Panda
    \item \textbf{Tracing:} LTTng via ros2\_tracing
\end{itemize}

\subsection{Experiment Design}

\subsubsection{Scenarios}

We evaluate three scenarios:

\begin{enumerate}
    \item \textbf{Baseline:} No artificial load
    \item \textbf{CPU Load:} stress-ng with varying intensity
    \item \textbf{Message Load:} High-frequency publishers
\end{enumerate}

\subsubsection{Metrics}

We measure the following metrics:

\begin{itemize}
    \item \textbf{T1:} Planning latency (goal to trajectory)
    \item \textbf{T2:} Execution latency (trajectory to completion)
    \item \textbf{T3:} Total end-to-end latency
    \item \textbf{T4:} Control loop jitter
    \item \textbf{T5:} State feedback delay
\end{itemize}

\subsection{Data Collection}

Each experiment trial follows this protocol:

\begin{enumerate}
    \item Start tracing session
    \item Apply load condition
    \item Execute pick-and-place task
    \item Stop tracing
    \item Extract and analyze traces
\end{enumerate}

% -----------------------------------------------------------------------------
% Results
% -----------------------------------------------------------------------------

\section{Results}
\label{sec:results}

\subsection{Baseline Performance}

% Include generated tables
% \input{analysis/output/tables/baseline_summary.tex}

\begin{table}[htbp]
\centering
\caption{Baseline Latency Statistics}
\label{tab:baseline}
\begin{tabular}{lrrrrr}
\toprule
Metric & Mean (ms) & Std (ms) & P95 (ms) & P99 (ms) & Max (ms) \\
\midrule
Planning (T1) & -- & -- & -- & -- & -- \\
Execution (T2) & -- & -- & -- & -- & -- \\
Total (T3) & -- & -- & -- & -- & -- \\
\bottomrule
\end{tabular}
\end{table}

\subsection{Load Impact Analysis}

\subsubsection{CPU Load Effects}

% Include generated figures
\begin{figure}[htbp]
    \centering
    % \includegraphics[width=0.8\textwidth]{analysis/output/plots/cpu_sweep.png}
    \caption{Latency vs CPU Load Percentage}
    \label{fig:cpu_sweep}
\end{figure}

\subsubsection{Message Load Effects}

% Similar structure for message load results

\subsection{Statistical Analysis}

% Include statistical comparison table
% \input{analysis/output/stats/stat_comparison.tex}

\begin{table}[htbp]
\centering
\caption{Statistical Comparison vs Baseline}
\label{tab:stats}
\begin{tabular}{lrrrr}
\toprule
Scenario & Change (\%) & t-statistic & p-value & Cohen's d \\
\midrule
CPU Load (50\%) & -- & -- & -- & -- \\
CPU Load (75\%) & -- & -- & -- & -- \\
Message Load & -- & -- & -- & -- \\
\bottomrule
\end{tabular}
\end{table}

\subsection{Breaking Point Analysis}

% Results from find_breaking_point.py

\subsection{Callback Chain Analysis}

% Results from e2e_path_analyzer.py

\begin{figure}[htbp]
    \centering
    % \includegraphics[width=0.9\textwidth]{analysis/output/paths/callback_heatmap.png}
    \caption{Callback Contribution Heatmap}
    \label{fig:callback_heatmap}
\end{figure}

% -----------------------------------------------------------------------------
% Discussion
% -----------------------------------------------------------------------------

\section{Discussion}
\label{sec:discussion}

\subsection{Key Findings}

\begin{enumerate}
    \item Finding 1: ...
    \item Finding 2: ...
    \item Finding 3: ...
\end{enumerate}

\subsection{Implications}

The results have several implications for \ros system design...

\subsection{Limitations}

This study has the following limitations:

\begin{itemize}
    \item Simulation-only experiments
    \item Single robot platform
    \item Limited load configurations tested
\end{itemize}

\subsection{Future Work}

Future research directions include:

\begin{itemize}
    \item Real hardware validation
    \item Multi-robot scenarios
    \item Adaptive load balancing strategies
\end{itemize}

% -----------------------------------------------------------------------------
% Conclusion
% -----------------------------------------------------------------------------

\section{Conclusion}
\label{sec:conclusion}

This work presented a comprehensive performance analysis of \ros manipulation stacks...

% -----------------------------------------------------------------------------
% Acknowledgments
% -----------------------------------------------------------------------------

\section*{Acknowledgments}

We thank [acknowledgments]. CloudLab resources were provided by [details].

% -----------------------------------------------------------------------------
% References
% -----------------------------------------------------------------------------

% \printbibliography

\begin{thebibliography}{9}

\bibitem{picas}
H. Choi et al., ``PiCAS: New Design of Priority-Driven Chain-Aware Scheduling for ROS2,'' \textit{RTAS}, 2021.

\bibitem{paam}
D. Casini et al., ``Response-Time Analysis of ROS 2 Processing Chains Under Reservation-Based Scheduling,'' \textit{ECRTS}, 2019.

\bibitem{ros2tracing}
C. Bedard et al., ``ros2\_tracing: Multipurpose Low-Overhead Framework for Real-Time Tracing of ROS 2,'' \textit{IEEE RA-L}, 2022.

\bibitem{moveit}
D. Coleman et al., ``Reducing the Barrier to Entry of Complex Robotic Software: a MoveIt! Case Study,'' \textit{J. Software Eng. Robotics}, 2014.

\end{thebibliography}

% -----------------------------------------------------------------------------
% Appendix
% -----------------------------------------------------------------------------

\appendix

\section{Experiment Configuration}
\label{app:config}

\begin{lstlisting}[language=yaml,caption={experiment\_config.yaml}]
experiment:
  num_trials: 30
  scenario: baseline

benchmark:
  task: pick_and_place
  timeout_s: 300

tracing:
  enabled: true
  session_name: ldos_trace
  events:
    - ros2:callback_start
    - ros2:callback_end
\end{lstlisting}

\section{Additional Figures}
\label{app:figures}

% Additional figures and tables

\section{Raw Data}
\label{app:data}

Complete experimental data is available at: \url{https://github.com/[repo]}

\end{document}
